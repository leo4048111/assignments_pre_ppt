\section{本次里程碑任务}
\begin{frame}
    \frametitle{本次里程碑任务}
    \begin{itemize}
        \item 使用基于Gradio + FastAPI的Web应用快速开发部署技术,以Restful接口文档为功能开发核心导向,重构项目AI绘图生成服务的底层代码架构与算法逻辑。采用完全自主实现的轻量级任务队列 + 悲观锁机制,实现Web系统高并发情景下的AI绘图服务高可用。
        \item 在迭代1实现的系统功能上继续增量开发,优化已有功能的运行性能和Web界面样式,同时增量地实现了画廊作品展示、用户反馈、img2img、快速检索等一系列功能需求
        \item 配置华为云提供的ECS弹性云服务器,确定项目软件系统的部署图,在弹性云服务器上部署项目软件系统,实现了项目软件系统的部署与运行
    \end{itemize}
\end{frame}

\begin{frame}
    \frametitle{本次里程碑任务}
    \begin{itemize}
        \item 根据甲方的SRS需求文档,为软件系统增添了后台管理系统,实现了对用户账号、模型、反馈等相关数据的管理(将简单演示)
        \item 完善了基于Apifox的接口管理规范、测试用例和Mock数据,实现更加健全完善的项目自动化测试体系和管理流程。
    \end{itemize}
\end{frame}

\begin{frame}
    \frametitle{工作项总览与相关图表}
    详见华为云
\end{frame}


\begin{frame}
    \frametitle{项目技术栈}
    \begin{itemize}
        \item 主要开发语言:Python
        \item HTTP服务器:Tornado + FastAPI
        \item 前端技术选型:Vue.js + Element UI/HTML + CSS + Js/Gradio
        \item 持久层框架:SQLAlchemy
        \item 数据库服务:MySQL + Redis
        \item 版本管理工具:Git
        \item 远程代码托管平台:华为云
        \item 接口管理与自动化测试工具:Apifox + Mock.js
    \end{itemize}
\end{frame}
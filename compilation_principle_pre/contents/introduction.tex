\section{课程设计内容}
\begin{frame}
    \frametitle{项目开发任务目标}
    \footnotesize
    \begin{itemize}
        \item {类C编译器程序架构设计与实现:使用高级程序语言作为实现语言,设计并实现一个类C语言的编译器,编码实现编译器的组成部分。}
        \item {词法分析任务:对于词法分析任务,给出类C语言的单词子集及机内表示,输入为源程序字符串,输出为单词的机内表示序列。}
        \item {语法分析任务:对于语法分析任务,通过LR(1)或者递归下降等语法分析方法设计并且构建语法分析器,同时在语法分析过程中一遍地调用词法分析器的nextToken方法获取下一个token,推进语法分析过程。}
        \item {中间代码生成任务:使用语法制导翻译技术,选择合适的中间代码表示形式(本项目中采用LLVM IR),要求能够在语法分析的同时生成中间代码,并且将生成结果保存到文件中。}
        \item {目标代码输出任务:编译器能够根据输入的类C语言源程序,还有运行时的参数选择,针对不同的处理器架构(target),输出例如x86,x86-64,mips等多种汇编代码,并且根据运行时参数进行不同等级的代码优化。生成后的汇编代码文件可以链接第三方编译环境提供的相关类库,进一步生成可执行文件。}
    \end{itemize}
\end{frame}

\begin{frame}
    \frametitle{项目开发任务目标}
    \footnotesize
    \begin{itemize}
        \item {实现过程、函数调用、指针、数组和GCC风格内联汇编的代码编译:本实验中,我已经实现了包括过程、函数调用(支持递归)、指针、数组等等各种类C语言的文法扩展和编译能力,使得编译器的功能更加健全与完备,具体实现的扩展功能点将在报告下文中进行详细阐述。}
    \end{itemize}
\end{frame}

\begin{frame}
    \frametitle{预备知识}
    \footnotesize
    \begin{itemize}
        \item {词法分析器设计原理}
        \item {文法分析方法,包括LR(1)分析方法、递归下降分析法,ACTION表和GOTO表的推导方法以及分析流程逻辑等等}
        \item {类C语言语法规范文法的设计}
        \item {语义分析与中间代码产生原理与技术}
        \item {LLVM IR Builder库编译、集成与部署技术,用以发射规范的LLVM IR中间代码,这种中间代码表示形式相比于四元式而言可读性和规范性更强,并且更加容易进行代码优化。}
        \item {中间代码优化与目标代码生成技术}
    \end{itemize}
\end{frame}

\begin{frame}
    \frametitle{开发环境}
    \footnotesize
    \begin{itemize}
        \item {OS:Windows 11 Pro/Mac OS/Ubuntu20.04(可跨平台)}
        \item {Language:CPP}
        \item {IDE:vscode + visual studio}
        \item {编译环境:Cmake + MinGW64 + MSVC + Clang}
        \item {报告绘图工具:StarUML + Doxygen}
        \item {LLVM版本:version 17.0.0 git Optimized build.}
    \end{itemize}
\end{frame}

\begin{frame}
    \frametitle{依赖库}
    \footnotesize
    \begin{itemize}
        \item {json.hpp (https://github.com/nlohmann/json):用于以json格式dump tokens}
        \item {rang.hpp (https://github.com/agauniyal/rang):用于修改控制台输出字体的颜色、样式等。}
        \item {LLVM version 17.0.0 (https://github.com/llvm/llvm-project):使用llvm::cl::opt处理程序运行时的启动参数,使用LLVM提供的LLVM IR Builder相关接口生成LLVM中间代码,使用LLVM Pass进行代码优化和目标代码生成。}
    \end{itemize}
\end{frame}
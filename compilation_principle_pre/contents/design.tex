\section{项目详细设计}
\begin{frame}
    \frametitle{类设计}
    \footnotesize
    \begin{itemize}
        \item {File类:用于储存读入的源文件内容,提供了操作文件的相关接口,包括数据获取、移动行号、信息记录等等}
        \item {Lexer类:词法分析器实现主体类,在一遍过程中为语法分析器提供nextToken()接口用于推进分析过程}
        \item {Parser和LR1Parser类:实现了递归下降和LR1分析两种语法分析方法的语法分析器实现主体类,输出相同的AST数据结构}
        \item {LLVMIRGenerator和IRGenerator类:分别使用语法制导翻译技术,输出LLVM IR和四元式形式表示的中间代码序列}
        \item {CodeGenerator类:使用LLVM IR中间代码序列作为输入,执行代码优化逻辑后输出可执行的目标文件}
        \item {相关工具类:包括错误处理、日志记录、运行时参数解析等于项目实现相关的工具类,可在源码中查看,此处不再赘述}
        \item {AST节点类:本实验中的AST节点类设计参考了Clang源码中的设计架构,为每种实现需求中的文法符号实现了对应的节点类型定义,此处重点阐述其中所有类型的具体含义}
    \end{itemize}
\end{frame}

\section{甲方在软件生命周期中的参与情况}
\begin{frame}
    \frametitle{软件定义时期}
    \footnotesize
    \begin{block}{甲方参与流程}
        本小组在PPT Copilot软件系统项目开发流程中,作为甲方参与项目,与乙方积极交流、沟通,合作,主要参与了软件生命周期中的以下的项目开展环节:
        \begin{itemize}
            \item 问题定义:本小组作为甲方,在项目开展初期对项目需求进行了详细的定义和规划,对软件的功能、性能需求以及目标用户进行了明确的设定。
            \item 可行性分析:本小组作为甲方,在项目开展初期对项目的可行性进行了分析,对软件的技术可行性、经济可行性、法律可行性以及进度可行性进行了详细的分析。
            \item 需求分析:本小组作为甲方,同样在项目开展初期明确了软件应用的场景,用户需求,核心产品痛点和主要抓手等需求内容,随后与乙方进行交流沟通,进一步调整并且完善软件系统需求内容,并且将这些需求清晰地向乙方表达。
            \item SRS规格说明文档:本小组作为甲方,将上述的三阶段分析内容整理为了详细的SRS文档,从而使得乙方能够更好地理解软件的需求,为软件的开发提供了明确的目标。
        \end{itemize}
    \end{block}
\end{frame}

\begin{frame}
    \frametitle{软件开发时期}
    \footnotesize
    \begin{block}{甲方参与流程}
        \begin{itemize}
            \item 系统部署:本小组作为甲方,根据乙方提供的部署操作说明,正确在ECS上部署并且运行了软件系统,以便后续测试工作的开展。
            \item 单元测试:本小组作为甲方,在乙方将系统源码发给我们后,简单分析了一下项目模块单元内部的实现代码架构与逻辑,随后主要采用白盒测试的相关覆盖方法进行了单元测试,从而保证了软件系统的基本单元功能的正确性。
            \item 综合测试:本小组作为甲方,采用等价类划分、场景法等黑盒测试方法,通过Apifox和乙方实现的Web端界面,合理设计测试用例,对于系统功能进行了比较全面且详细的测试,并且将测试结果向乙方反馈,进一步进行沟通交流,从而保证了软件系统的功能性、可用性、易用性、稳定性等质量属性。
            \item 测试报告:本小组作为甲方,对于软件系统的测试结果进行了详细的整理和总结,撰写了详细的测试报告,从而使得乙方能够更好地理解软件系统的测试情况,为软件的上线提供了明确的目标。
        \end{itemize}
    \end{block}
\end{frame}

\begin{frame}
    \frametitle{软件维护时期}
    \footnotesize
    \begin{block}{甲方参与流程}
        \begin{itemize}
            \item 系统维护:TODO
            \item 系统重构:TODO
        \end{itemize}
    \end{block}
\end{frame}






%% introductioin.tex
%% Copyright 2022 skyleaworlder
%
% This work may be distributed and/or modified under the
% conditions of the LaTeX Project Public License, either version 1.3
% of this license or (at your option) any later version.
% The latest version of this license is in
%   http://www.latex-project.org/lppl.txt
% and version 1.3 or later is part of all distributions of LaTeX
% version 2003/12/01 or later.
%
% This work has the LPPL maintenance status "maintained".
%
% This Current Maintainer of this work is skyleaworlder.
%
% This work consists of all the *.tex and *.sty files in
%   https://github.com/TJ-CSCCG/Tongji-Beamer
\section{SRS需求文档修订}
\begin{frame}
    \frametitle{主功能模块}
    \footnotesize
    \begin{block}{修改内容}
        本小组在需求分析阶段,与乙方积极沟通,对于如下SRS需求文档中的主功能模块进行了修改:
        \begin{itemize}
            \item 账号管理:简化了账号管理模块功能,移除了用户充值、用户权限管理等部分用例
            \item 用户充值:完全移除了用户充值功能模块需求,将用户充值功能作为后续的扩展功能进行考虑
            \item 内容自动化生成:简化了ChatGPT对接逻辑,删除了关于Markdown中间件生成的用例,由乙方在详细设计中自行决定如何通过ChatGPT驱动PPT文档内容生成
            \item 项目管理:删除了部分项目管理用例,使得项目管理模块更加简洁
        \end{itemize}
    \end{block}
\end{frame}

\begin{frame}
    \frametitle{其它需求}
    \footnotesize
    \begin{block}{修改内容}
        \begin{itemize}
            \item 性能需求:降低了对于系统并发性能、吞吐量的要求
            \item 质量属性需求:降低了对于系统的可用性、易用性、可靠性、可维护性、可移植性的要求,使得系统更加注重功能性
        \end{itemize}
    \end{block}
\end{frame}




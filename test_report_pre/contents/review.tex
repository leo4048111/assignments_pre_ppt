\section{迭代三反思}
\begin{frame}
    \frametitle{目标和预期}
    \begin{itemize}
        \item 围绕甲方SRS需求文档为核心,高质量地完成相关功能需求的开发任务
        \item 围绕功能用例,时刻开展自动化测试,保证项目的稳定性和可用性
        \item 上线项目,邀请用户体验并给出测试反馈,收集用户意见和Bug反馈,为第三次迭代的内容规划提供参考
    \end{itemize}
\end{frame}

\begin{frame}
    \frametitle{信息和工具}
    \begin{itemize}
        \item 小组中对于相关功能需求分析实现,很大程度上参考了甲方提供的SRS需求文档,以及甲方提供的用例图,从而使得设计与实现内容与甲方需求最大程度地做到匹配
        \item 技术上,我们小组借助相关网络资料以及以往的开发经验,并且和相关从业专业人士进行沟通,充分对于当前行业的流行技术生态进行调研,选择最有利于敏捷开发的技术栈,从而使得项目开发效率得到了很大的提升
    \end{itemize}
\end{frame}

\begin{frame}
    \frametitle{困难与阻碍}
    \begin{itemize}
        \item 我们小组开发在迭代三遇到的最大的困难是在项目部署到服务器运行并测试时,遇到项目在实际的网络环境与本地环境下运行的差异,导致项目之前的许多功能实现可能在本地运行高效、良好,但是在网络环境下则由于网络带宽、高并发场景等原因,导致项目运行效率低下。这导致我们虽然很快地实现了相关的项目需求,但是依然需要负责相关功能模块的同学花费许多的时间对于原本功能实现的底层算法与代码逻辑进行优化,尽可能地降低算法对于系统资源的消耗。这也使得我们深刻意识到了实际生产环境下的项目开发与测试的重要性,以及对于项目的高效性、可用性的重视程度,让我们对于实际可用的工业化项目的敏捷开发流程有了更加深刻的认识。
    \end{itemize}
\end{frame}

\begin{frame}
    \frametitle{优势和创新}
    \begin{itemize}
        \item 得益于我们小组在迭代1中搭建的良好项目管理架构,我们在迭代三中进一步对于整个管理流程进行了完善,高度规范化了组员从功能开发、自动化测试、代码提交、审查、合并、部署、上线等整个项目开发流程,从而使得我们小组的开发人员能够更加专注于项目的功能实现,而不是处理各种由于项目流程管理不当产生的问题,使得我们小组的开发效率得到了进一步的提升。
        \item 我们小组在本次迭代中,着重进行了后端服务的相关优化升级,对于相关的服务代码架构进行了重构与测试,成功在高并发的情景下,保证了项目的高效稳定运行,使得我们小组的项目更加具有可用性和可靠性。
    \end{itemize}
\end{frame}

\begin{frame}
    \frametitle{结果和进度}
    \begin{itemize}
        \item 我们小组在每周的组会上,都会各自汇报本周自主开发以及集中开发中完成任务的情况,并且由会议记录人员进行统计。最终结果表明,小组在本次迭代中,完成了甲方提出的所有功能需求,并且在自动化测试方面也取得了很大的进展,使得我们小组的项目更加稳定可靠,结果还是比较让我们感到欣喜和自豪的。
    \end{itemize}
\end{frame}

\begin{frame}
    \frametitle{情绪状态}
    \begin{itemize}
        \item 我们小组成员在本次迭代开发过程中,都非常重视开发工作,花费了大量时间完成相关工作项。期间许多组员感染新冠阳性,但依然带病线上参与开发工作,讨论项目进度,完成相关任务,体现了我们小组成员的责任心和敬业精神,也使得我们小组的项目开发进度得到了保证。
    \end{itemize}
\end{frame}

\begin{frame}
    \frametitle{心得体会与开发意义}
    \begin{itemize}
        \item 通过本次迭代任务,我们小组成员之间的配合更加融洽并且具有默契,能够作为一个团队更加高效与良好地完成相关的项目开发任务,并且更加熟悉了类似的大型项目的开发流程与管理模式,为之后的下一次迭代乃至今后学习工作中的类似项目开发情景打下了良好的基础。
    \end{itemize}
\end{frame}
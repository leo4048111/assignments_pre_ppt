%% introductioin.tex
%% Copyright 2022 skyleaworlder
%
% This work may be distributed and/or modified under the
% conditions of the LaTeX Project Public License, either version 1.3
% of this license or (at your option) any later version.
% The latest version of this license is in
%   http://www.latex-project.org/lppl.txt
% and version 1.3 or later is part of all distributions of LaTeX
% version 2003/12/01 or later.
%
% This work has the LPPL maintenance status "maintained".
%
% This Current Maintainer of this work is skyleaworlder.
%
% This work consists of all the *.tex and *.sty files in
%   https://github.com/TJ-CSCCG/Tongji-Beamer
\section{项目内容简介}
    \begin{frame}
    % “无序列表” 与 “有序列表” 使用
    \frametitle{项目内容简介}
        \footnotesize
        \begin{block}{项目开发背景}
            \begin{itemize}
                \item {随着人工智能技术的飞速发展,AI绘画技术也日趋成熟。近些日子来,无数精美的AI绘画作品都让我们眼前一新。对于没有接触过AI绘画领域的小白,他们可能也想体验AI绘画的奇妙,但不知从何下手。对于钻研AI绘画领域的技术人员,他们可能想分享自己的训练成果,同时和其他从业人员沟通交流,但是缺乏相关的平台。本软件的创建便是为了解决以上问题,为小白和技术人员提供一个在线生成图片,上传分享训练参数并和他人沟通的平台。}
            \end{itemize}
        \end{block}

        \begin{block}{项目核心功能需求}
            \begin{enumerate}
                \item 用户能够在线欣赏、生成图片,下载相关模型参数
                \item 用户能够上传分享自己训练的模型参数
                \item 用户能够使用软件系统提供的相关社交功能,与其他用户进行交流
            \end{enumerate}
        \end{block}
    \end{frame}


\section{本次里程碑任务}
\begin{frame}
    \frametitle{本次里程碑任务}
    \begin{itemize}
        \item 根据上述issue抽取工作中的相关内容项,对于项目开发中的技术债进行偿还,进一步提升系统的运行性能和可用性
        \item 进一步部署并且完善项目的自动化测试和CI流程,通过Apifox管理开发工作流,部署整套的项目自动化集成与测试解决方案,并且撰写测试报告与系统文档,为项目交付上线做好准备
        \item 使用docker image打包环境,并且编写详细的测试报告、软件部署说明和软件使用说明,尽可能降低甲方的部署操作成本,准备软件系统交付工作
        \item 按照小组开发章程,定时开展组会进行小组内部沟通,拉通对齐组员开发进度,稳步推进开发工作,保证在项目开发后期能够按照计划完成项目开发任务,交付能够让甲方客户满意的软件系统产品
    \end{itemize}
\end{frame}

\begin{frame}
    \frametitle{工作项总览与相关图表}
    详见华为云
\end{frame}

\begin{frame}
    \frametitle{项目技术栈}
    \begin{itemize}
        \item 主要开发语言:Python
        \item HTTP服务器:Tornado + FastAPI
        \item 前端技术选型:Vue.js + Element UI/HTML + CSS + Js/Gradio
        \item 持久层框架:SQLAlchemy
        \item 数据库服务:MySQL + Redis
        \item 版本管理工具:Git
        \item 远程代码托管平台:华为云
        \item 接口管理与自动化测试工具:Apifox + Mock.js
    \end{itemize}
\end{frame}
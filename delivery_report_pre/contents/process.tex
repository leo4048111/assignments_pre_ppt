\section{开发过程}
\begin{frame}
    \frametitle{敏捷方法框架}
    采用scrum敏捷方法框架指导开发流程,遵循下列开发原则
    \begin{itemize}
        \item 快速反馈:一般1-2周一个迭代周期,也是一个反馈周期
        \item 尽早交付:高优先级需求及时满足
        \item 适应变化:小步快跑,不断修正
        \item 持续改进:不断反思、回顾、优化
        \item 客户满意:持续在每个里程碑结束时进行进度汇报,与甲方保持沟通,不断反馈修正需求
    \end{itemize}
\end{frame}

\begin{frame}
    \frametitle{Milestone 1}
    经过对于小组开发流程文档记录的回顾和整理,我们小组对于各个里程碑中完成的工作进行了总结和反思,总结如下:
    \begin{itemize}
        \item 确定项目开发技术栈,搭建项目开发环境,使得相关框架能够正确集成并且运行。
        \item 根据甲方项目需求 SRS 文档,完成项目的需求分析,确定项目的主要功能,完成项目的功能设计与主要功能模块划分。
        \item 完成项目软件架构的逻辑设计和物理设计,主要包括数据库schema 设计、索引设计、存储过程设计和视图设计,基于Restful API 接口规范进行后端接口路由和功能设计等。
        \item 基于上述设计内容,使用华为云平台进行版本管理,将大功能模块设置为 EPIC 工作项,并且在 Epic 下抽取 Feature,对于每个 Feature 进一步划分 User Story,并且确定第一次迭代中需要实现的用户故事,最后分配工作到每位组员进行代码开发。
    \end{itemize}
\end{frame}

\begin{frame}
    \frametitle{Milestone 2}
    \begin{itemize}
        \item 使用基于Gradio + FastAPI的Web应用快速开发部署技术,以Restful接口文档为功能开发核心导向,重构项目AI绘图生成服务的底层代码架构与算法逻辑。采用完全自主实现的轻量级任务队列 + 悲观锁机制,实现Web系统高并发情景下的AI绘图服务高可用。
        \item 在迭代1实现的系统功能上继续增量开发,优化已有功能的运行性能和Web界面样式,同时增量地实现了画廊作品展示、用户反馈、img2img、快速检索等一系列功能需求
        \item 配置华为云提供的ECS弹性云服务器,确定项目软件系统的部署图,在弹性云服务器上部署项目软件系统,实现了项目软件系统的部署与运行
    \end{itemize}
\end{frame}

\begin{frame}
    \frametitle{Milestone 2}
    \begin{itemize}
        \item 根据甲方的SRS需求文档,为软件系统增添了后台管理系统,实现了对用户账号、模型、反馈等相关数据的管理
        \item 完善了基于Apifox的接口管理规范、测试用例和Mock数据,实现更加健全完善的项目自动化测试体系和管理流程。
    \end{itemize}
\end{frame}

\begin{frame}
    \frametitle{Milestone 3}
    \begin{itemize}
        \item 根据之前里程碑汇报中与甲方的沟通结果进行issue抽取,并且分配工作项到组员,对于项目开发中的技术债进行偿还,进一步提升系统的运行性能和可用性
        \item 进一步部署并且完善项目的自动化测试和 CI 流程,通过Apifox 管理开发工作流,部署整套的项目自动化集成与测试解决方案,并且撰写测试报告与系统文档,为项目交付上线做好准备
        \item 使用 docker image 打包环境,并且编写详细的测试报告、软件部署说明和软件使用说明,尽可能降低甲方的部署操作成本,准备软件系统交付工
        \item 按照小组开发章程,定时开展组会进行小组内部沟通,拉通对齐组员开发进度,稳步推进开发工作,保证在项目开发后期能够按照计划完成项目开发任务,交付能够让甲方客户满意的软件系统产品
    \end{itemize}
\end{frame}


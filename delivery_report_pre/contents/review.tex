\section{系统开发反思}
\begin{frame}
    \frametitle{系统安全性}
    \begin{itemize}
        \item 对于密码存储采用了Sha256Hash+salt的加密存储方式,保证了用户密码的安全性
        \item 对于HTTP请求进行用户会话验证,保证了用户的安全性
        \item 对于用户输入参数从前端和后端两个方面进行了校验,防止了SQL注入、XSS攻击的风险,保证了系统的安全性
        \item 设定ECS端口出入规则,保证了系统安全性
        \item js等项目代码未进行混淆,存在源码泄露的风险
    \end{itemize}
\end{frame}

\begin{frame}
    \frametitle{系统性能}
    \begin{itemize}
        \item 系统在正常负载下运行流畅,响应时间快,符合用户体验需求
        \item 通过使用缓存技术,引入Redis中间件隔离数据库事务,减少了数据库的访问频率,从而提高了系统性能
        \item 但在并发量较高的情况下,系统的响应速度有所下降,可能需要进行负载均衡优化
        \item 数据库表的设计还可以更加合理化,从而加快系统的查询事务速度
    \end{itemize}
\end{frame}

\begin{frame}
    \frametitle{流程管理}
    \begin{itemize}
        \item 在项目初期,我们就明确了整个开发的流程,包括需求分析、设计、开发、测试和维护等阶段,提供了对整个项目进度的可控性
        \item 通过使用华为云进行项目管理,通过图表和平台提供的相关工具可视化管理流程,使得任务分配和进度追踪更为明确,效率大大提高
        \item 使用了版本控制系统Git,部署了完整的代码提交、审核门禁与分支管理系统,方便了多人协作和版本追溯
        \item 在需求变更管理上,我们也严格控制了需求的变动,减少了不必要的开发工作
        \item 在单元测试上,我们采用了覆盖方法,但是并没有非常完整地覆盖所有条件组合,使得在后续系统测试中,甲方依然在我们单元测试通过的模块中检测出了一些缺陷
    \end{itemize}
\end{frame}

\begin{frame}
    \frametitle{组员分工情况}
    \begin{table}[]
        \centering
        \resizebox{\textwidth}{!}{
            \begin{tabular}{c|c|c}
                \textbf{组员信息} & \textbf{主要工作项}                                    & \textbf{工作量占比} \\ \hline
                2051857 曾诗容    & 需求分析、文档撰写、编码工作                           & 16.71\%             \\ \hline
                2052636 陈骁      & 系统架构设计、数据库设计、云服务器部署与维护、编码工作 & 17.14\%             \\ \hline
                2050250 李其桐    & 前端设计、前端代码实现、后端接口的实现                 & 17.27\%             \\ \hline
                2054080 林奕如    & 后端主体代码的编写、单元测试的编写、代码评审           & 16.86\%             \\ \hline
                2053865 刘昱彤    & 项目管理、沟通协调、后端接口的实现                     & 16.71\%             \\ \hline
                1751118 吴达鹏    & 系统安全性分析、性能优化、云服务器部署与维护、编码工作 & 17.54\%             \\ \hline
                2053868 于采篱    & 系统测试、BUG修复、编码工作                            & 16.43\%             \\
            \end{tabular}
        }
        \caption{工作量表}
        \label{tab:tech-strategy}
    \end{table}
    定量衡量标准:
    \begin{itemize}
        \item 每周例会工作进度统计情况、git commit记录、华为云工作项分配与完成情况、文档撰写字数提交记录
    \end{itemize}
\end{frame}

\begin{frame}
    \frametitle{总结}
    \begin{itemize}
        \item 通过这个项目的开发,我们团队得到了非常宝贵的经验。我们认识到了对软件开发全周期的理解和遵循的重要性,对安全性、性能等关键因素的重视,对团队分工和管理的合理性和高效性的追求。
        \item 在流程管理方面,我们应该更严谨的控制需求变更,并提高单元测试的覆盖率,降低后续测试中出现的问题和漏洞。
        \item 今后的项目开发中,对于团队分工和协作,我们将更注重均衡性和协调性,让每个成员的工作量更加公平,让团队的工作效率和效果更好。
        \item 项目开发流程中,我们与甲方同学的沟通非常密切、积极并且愉快,最终交付的软件系统产品也得到了甲方同学的认可,我们对此感到非常高兴。
    \end{itemize}
\end{frame}

\begin{frame}
    \frametitle{项目开源}
    本项目源码已经使用{AGPL-3.0}开源协议,开源在了{https://github.com/leo4048111/VisionaryArt-mirror}下
\end{frame}


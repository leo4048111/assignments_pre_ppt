%% introductioin.tex
%% Copyright 2022 skyleaworlder
%
% This work may be distributed and/or modified under the
% conditions of the LaTeX Project Public License, either version 1.3
% of this license or (at your option) any later version.
% The latest version of this license is in
%   http://www.latex-project.org/lppl.txt
% and version 1.3 or later is part of all distributions of LaTeX
% version 2003/12/01 or later.
%
% This work has the LPPL maintenance status "maintained".
%
% This Current Maintainer of this work is skyleaworlder.
%
% This work consists of all the *.tex and *.sty files in
%   https://github.com/TJ-CSCCG/Tongji-Beamer
\section{引言}
    \begin{frame}
    % “无序列表” 与 “有序列表” 使用
    \frametitle{引言}
        \footnotesize
        \begin{block}{项目提出背景}
            \begin{itemize}
                \item 	当前,ChatGPT已经发布并且对公众开放了服务接口,这无疑标志着一个人工智能的新纪元已然到来。通过ChatGPT的强势赋能,使得许多传统工作流都得到了极大的颠覆与创新,在达到更高效率的同时也能够确保质量。本项目正是在这一背景下,基于ChatGPT的公开API接口,搭建一个能够根据用户的自然语言描述需求,自动生成Markdown格式文档,然后通过Markdown解析器处理文件文本内容从而生成ppt的软件系统,从而能够在人们的实际文档设计与编写工作中,以一个可靠软件助手的姿态提供辅助,有效提升人们的工作效率。
            \end{itemize}
        \end{block}

        \begin{block}{项目意义}
            \begin{enumerate}
                \item 为响应在互联网传统工作方式中,企业内部、学生和个人对PPT文档的自动化生成和在线编辑需求而进行设计和开发。
                \item 为用户节省编辑成本,提升编辑效率,拥有广泛的应用前景。
                \item 对于其它竞品的相关功能特性进行研究分析,并且在其基础上进行精炼、完善,同时围绕核心业务设计并且实现额外的使用子功能模块。
            \end{enumerate}
        \end{block}
    \end{frame}


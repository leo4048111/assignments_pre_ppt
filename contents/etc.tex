\section{非功能需求}
    \begin{frame}
        \frametitle{质量属性}
        \begin{itemize}
            \item 可用性:一般情况下,要求99\% 以上的时间可用,在工作日工作时段,考虑到可能会由于流量较大产生拥挤,要求系统95\%以上时间可用。特殊情况下,即若登录频率超过 500次/小时,要求 85\% 以上的时间可用。
            \item 可扩展性:系统实现工程必须遵循良好的架构规范,从而能够支持并行开发,使得功能扩展能够仅通过添加相关程序文件、简单地增加接口、简单修改外部界面实现。
            \item 安全性:要求要对外部和内部的网络攻击有较好的防范能力,并且在安全漏洞出现时,要有较快的修复响应能力。同时,应当对于用户的具体类型,实施严格的权限管理,对于用户操作需要进行session验证和验签,防止通过直接请求接口或者路由就能越权访问系统服务的情况发生。
        \end{itemize}
    \end{frame}

    \begin{frame}{质量属性}
        \begin{itemize}
            \item 可靠性:要求系统在正常操作条件下,能够持续运行24小时不出现故障。在系统出现故障时,要求在可接受的时间内进行恢复。同时,系统应当对于异常有良好的处理机制,避免由于用户的不当操作而导致后台系统崩溃。在系统不完全退出、断电等情况发生后,应当能够正确进行状态恢复,对于不完全的操作进行回滚,比如清除上传不完全的文件等,对于已经交易完毕但是没有来得及更新用户状态的订单进行处理等等。
            \item 互操作性:导出的相关文件和url应当遵循各自的格式规范,使得用户能够通过本地工具直接打开进行编辑。
            \item 可维护性:软件运行过程中中出现异常的点均应保存在一个仅由维护者可查看的文件里。
            \item 可重用性:能够在任意Windows系列和主流的GNU/Linux发行版操作系统上快速部署。
            \item 可测试性:对任意一个功能模块均能够单独测试,而不需要其他功能模块的支持。
        \end{itemize}
    \end{frame}

\section{非功能需求}
    \begin{frame}{质量属性}
        \begin{itemize}
            \item 可用性:保证高可用
            \item 可扩展性:功能扩展可仅通过添加文件、接口和修改外部界面实现
            \item 安全性:防范网络攻击
            \item 可靠性:可恢复、异常处理、不完全操作回滚
            \item 互操作性:导出文档能够直接在本地打开编辑
            \item 可维护性:用户操作记录、错误记录、网络访问记录等内容写入日志并落盘
            \item 可重用性:跨平台快速部署
            \item 可测试性:可进行单元测试、集成测试、系统测试、验收测试
        \end{itemize}
    \end{frame}

    \begin{frame}
        \frametitle{日志需求}
        \begin{enumerate}
            \item 后台应当实时维护能够让系统管理员查看的用户操作日志,其中记录包括用户行为、系统运行错误、网络访问记录和文件访问记录等主要内容。
            \item 用户行为中,登录、登出、注册、访问网页等主要行为需要被记录。
            \item 对存在可疑用户行为,比如对于登录频率超过20次/分钟、登录时连续输错密码超过10次,短时间类大量发起支付订单等行为的用户进行一定时间的禁用,并单独将行为记录在日志中。
            \item 日志数据应当持久化存储在磁盘上,并且支持快速记录检索和以文本形式导出并查看。
        \end{enumerate}
    \end{frame}

    \begin{frame}
        \frametitle{界面需求}
        \begin{enumerate}
            \item 软件系统界面元素应能够清晰呈现每个功能模块的用户交互逻辑,同时应避免过度设计,做到美观简洁。
            \item 软件系统界面的不同页面设计风格尽量统一。
        \end{enumerate}
    \end{frame}

    \begin{frame}
        \frametitle{服务器恢复需求}
        \begin{enumerate}
            \item 软件系统需要在例如物理机故障、掉电或者手动关闭服务时,能够尽可能保证关键数据,例如用户支付订单信息等的无差错和不丢失。
            \item 软件系统需要有定时备份机制,能够保证服务器上的数据可以定时备份,例如每周、每日或者每小时备份一次。 
            \item 软件系统需要有一个完整并且可操作性的恢复方案,可以表述在软件系统文档中,以能够在实际系统发生故障时,指导维护人员进行恢复操作。
        \end{enumerate}
    \end{frame}


